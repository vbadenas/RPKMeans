In this work we presented an implementation of the PRKM algorithm proposed by Marco Capó and Aritz Pérez and Jose A. Lozano in their paper \cite{CAPO201756} An efficient approximation to the K-means clustering for massive data. We have demonstrated though the experiments that the paper proposed that the solution implemented in this work is comparable to the implementation of the authors. We also reached results comparable to the ones that they achieved in their paper and corroborated the results and conclusions of their paper.

The RPKM algorithm is a good alternative to scikit-learn's KMeans and MiniBatchKMeans when dealing with large ammounts of data and with large ammounts of features as they reduce drastically the number of distance computations while obtaining a good aproximation of the cluster error. We believe that this algorithm has potential to be used efficiently when clustering large ammounts of data. This afirmation is backed by the results obtained in the experiments.